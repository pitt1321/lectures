%\documentclass{beamer}

\mode<presentation>{\usetheme{Goettingen}}

\usepackage[latin1]{inputenc}
\usepackage{listings}
\usepackage{times}
%\usepackage[T1]{fontenc}
% Or whatever. Note that the encoding and the font should match. If T1
% does not look nice, try deleting the line with the fontenc.


\title[PHYS 1321] % (optional, use only with long paper titles)
{Computational Methods in Physics}

\subtitle{PHYS 1321: Notes and Homework}

\author[] % (optional, use only with lots of authors)
{Prof. Michael Wood-Vasey}

\institute{University of Pittsburgh\\ Department of Physics and Astronomy}

\date[Week 1]{Week 1}


\begin{document}
\lstset{language=Python, basicstyle=\footnotesize\ttfamily}

\begin{frame}
  \titlepage
\end{frame}

\section<article>{PHYS 1321: Notes and Homework \hfill Week 1}
\subsection<article>{Computational Methods in Physics}
\mode<article>{\vspace{3mm} \hrule \vspace{5mm}}


\section{Course Information}

%\subsection[First Subsection Name]{First Subsection Name}


\begin{frame}{Course Structure}
  Syllabus at \url{http://github.com/pitt1321/syllabus}
  \begin{itemize}
    \item Organized by week:
    \begin{itemize}
      \item Lecture and lab time each Monday and Wednesday.
      \item More focus on lab time on Friday.\\ Bring questions!
      \item Assignments due Friday.  What time do you want?
    \end{itemize}
    \item Lecture and lab time will be interactive:
    \begin{itemize}
      \item Introduction of new material.
      \item Interactive demonstrations.
      \item Start your assignment.
      \item I will clarify topics as needed.
      \item Try not to miss classes; they will be hard to make up.
    \end{itemize}
    \item Computers options:
    \begin{itemize}
      \item Use lab computers with pre-installed Python and 
      \begin{itemize}
        \item Clone and commit/push back to GitHub for each session.
        \item store your data on a flash drive.
        \item You can try Box, but sadly Box doesn't always work the best with git.
        \item Use your own computer and install software yourself.
        \item I will provide some support, but your colleagues and the wider web will be the best resources.
      \end{itemize}  
    \end{itemize}  
  \end{itemize}
\end{frame}


\begin{frame}{Course Structure, Continued}
  \begin{itemize}
    \item All course materials available on GitHub page: \url{https://github.com/pitt1321/}
    \item Assignments:
    \begin{itemize}
      \item Each assignment will be provided as a GitHub repository.
% Under here:
%         \url{https://github.com/pitt1321/assignments/}
% Or perhaps assignment-by-assignment
      \item Fork the repository on GitHub from your own account
      \item ``git clone'' your fork
      \item Work on assignments in lab and on your own time.
      \item Regularly commit and push your work.
      \item Discussion is encouraged,\\ but all code entry must be your own!
      \item Create ``Issues'' to ask for help or feedback or just to track your own progress.
      \item Understand all the code you write!
      \item Submit assignment through a GitHub ``Pull Requests'' on your fork.
    \end{itemize}
    \item Special project:
    \begin{itemize}
      \item Final assignment of the semester.
      \item Start looking for a topic early!
      \item I will help with coding in lab time.
    \end{itemize}  
  \end{itemize}
\end{frame}


\begin{frame}{Where to Go for Help}  
\begin{itemize}
\item I will be around during ``working time'' during each class after lecture time.
\item Office hours: Wednesday, 11:00 to 12:00\\
\item Make an appointment to meet with me.
\item Lots of Python and GitHub help available online!
\end{itemize}
\end{frame}


\begin{frame}{Course Outline}
  \begin{itemize}
  \item Week 1: Introduction to Python.
  \item Week 2: Python arrays and plotting.
  \item Week 3: Random numbers, Monte Carlo simulation.
  \item Week 4: Numerical integration.
  \item Week 5: Solving equations, root finding.
  \item Week 6: NumPy and SciPy, matrix methods.
  \item Week 7: Ordinary differential equations (ODEs).
  \item Week 8: Solving systems of ODEs.
  \item Week 9: Fourier Transforms.
  \item Week 10: Discrete/continuous nonlinear problems.
  \item Week 11: Thermodynamic simulations.
  \item Week 12: Partial differential equations (PDEs).
  \item Week 13: PDEs continued.
  \item Week 14: Final project presentations.
  \end{itemize}
\end{frame}


\section{Introduction to Python}

\begin{frame}{What is Python?}
  \begin{itemize}
  	\item A high-level language.
  	\begin{itemize}
        \item Built-in high level data structures.
        \item Object oriented with some inspiration from functional languages.
  	\end{itemize}
    \item An interpreted language.
  	\begin{itemize}
        \item You don't compile your programs.
        \item Exception framework with tracebacks
        \item Automatic memory management
        \item Dynamic typing, dynamic binding.
  	\end{itemize}
    \item Huge standard library with all sorts of functionality.
    \item Extensible and embeddable.
    \item Cross-platform and free.
    \item Great as both a scripting/glue language and for full-blown application development.
  \end{itemize}
\end{frame}


\begin{frame}
\frametitle{Installation}
\begin{itemize}
  \item Lab computers have the software pre-installed.  If you want to use your personal machine, I recommend the Anaconda Python distribution:
  \begin{itemize}
    \item \url{https://store.continuum.io/cshop/anaconda/}
    \item This is a distribution aimed at the Scientific Python users, which is us!
    \item It includes the key main packages ```numpy```, ```scipy```, ```matplotlib```, and ```iPython```; and also provides the increasingly popular ```pandas``` data processing framework.
  \end{itemize}
  \item The Entought Canopy distribution is also a popular one
  \begin{itemize}
    \item Pros: 
    \begin{itemize}
      \item Comes with an Integrated Development Environment (IDE) which some people really like.  
      \item It's what is actually installed on the Thaw 210 lab computers.
    \end{itemize}
    \item Cons: 
    \begin{itemize}
      \item Is currently restricted to Python 2.7.
      \item Is nominally not free, but is free for academic use (i.e., for .edu email addresses).
    \end{itemize}
  \end{itemize}
\end{itemize}
\end{frame}


\begin{frame}[fragile=singleslide]
\frametitle{Running Python}
There are many ways Python can be used:
\begin{itemize}
  \item Interactively:
  \begin{itemize}
    \item Run the python program with no arguments and get:
  \end{itemize}
\begin{lstlisting}
Python 2.7.10 (default, Jul 24 2015, 10:36:25) 
[GCC 4.2.1 Compatible Apple LLVM 6.1.0 (clang-602.0.53)] on darwin
Type "help", "copyright", "credits" or "license" for more information.
>>> 
\end{lstlisting}
  \begin{itemize}
    \item Useful for tests, debugging, and for demonstrations.
  	\item This is where we'll start today.
  \end{itemize}
  \item Non-interactively:
  \begin{itemize}
    \item Write a script (a text file) and run it.
  	\item We will not use this, but it may be useful.
  \end{itemize}
  \item Using iPython Notebook:
  \begin{itemize}
    \item Present code, text, and results in one file.
    \item Allows interactive analysis
    \item Encourages integration of explanation, code, and results.
  	\item This is where we'll end up.
  \end{itemize}
\end{itemize}
\end{frame}


\begin{frame}[fragile=singleslide]
\frametitle{Introduction to iPython/Jupyter}
\begin{itemize}
  \item iPython is a useful and improved interface over the base python interactive command line.
  \item iPython Notebook is a web-browser-based interface that uses ``Notebook''s to present content, code, and results in an integrated file.
  \item From the command line type the following to open up a web browser that you will use to interact with iPython.
% \begin{lstlisting}
% ipython notebook
% \end{lstlisting}
  \begin{itemize}
    \item Main site: \url{http://ipython.org/}
    \item Quick demo: \url{https://youtu.be/H6dLGQw9yFQ}
    \item It's nice to see code and results together when developing
    \item The integrated help and auto-completion is really useful.
    \item Can save the notebook including results
    \item Can export the code to a separate Python file for reuse by other programs.
    \item Lots of videos and demos, including at \url{http://ipython.org/videos.html}.  Check them out and come up to speed.
    \item Tutorials and example notebook:  \url{http://nbviewer.ipython.org/github/ipython/ipython/blob/master/examples/Notebook/Index.ipynb}
    \item ``A gallery of interesting iPython Notebooks'' \url{https://github.com/ipython/ipython/wiki/A-gallery-of-interesting-IPython-Notebooks}
    \item NBViewer:  A service that you can publish read-only versions of iPython notebooks.  \url{nbviewer.ipython.org}
    \item More later.
  \end{itemize}
\end{itemize}
\end{frame}


\section{Homework}
\begin{frame}{Homework 1}{Due 2015-09-05, at TIME WE AGREED}
\begin{enumerate}
  \item Install Python and iPython on your computer.
  \begin{enumerate}
    \item Python version 3.4 or later.
    \item If you are already up and running with Python 2.7, you can keep running that but please be aware of the differences and submit code that runs under Python 3 using the ``\_\_future\_\_`` module: \url{https://docs.python.org/2/library/\_\_future\_\_.html}
  \end{enumerate}
  \item Complete Problem Set 1. Turn in your code via GitHub.
  \begin{enumerate}
    \item An iPython file with:
    \begin{enumerate}
      \item Your written answers.
      \item Your code.
      \item Your output.
      \item Images of plots as required.
    \end{enumerate}
  \end{enumerate}
\end{enumerate}
\end{frame}


\section{Python Basics}
% Insert external slides into pdf here


%\end{document}


